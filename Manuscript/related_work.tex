This Chapter highlights related work, especially that dealing with container cluster migration, 
software load balancer containerization, load balancer tools within the context of the container technology and scalable load balancer in the cloud providers.


This work is 
portable web apps
container cluster
orchestrator as a middleware
ingress routing using load balancer

\section{Portability of web applications}

\paragraph{\bf TOSCA:}


\paragraph{\bf Kubernetes federation:}
\added[id=2nd]{
Kubernetes developers are trying to add federation\cite{K8sFederation2017} capability for handling situations 
where multiple Kubernetes clusters\footnote{The {\em Kubernetes cluster} refers to a server cluster 
controlled by the Kubernetes container management system, in this paper.} 
are deployed on multiple cloud providers or on-premise data centers, 
and are managed via the Kubernetes federation API server (federation-apiserver). 
However, how each Kubernetes cluster is run on different types of cloud providers
and/or on-premise data centers, especially when the load balancers of such environments are not supported by Kubernetes, 
seems beyond the scope of that project. 
}

These work are not specific for web apps and lacks important features for them.

\section{Ingress routing in container orchestrator}
\added[id=2nd]{
Docker/Swarm
Mesos/Marathon
Kubernetes
}

\added[id=2nd]{
The swarm mode of the Docker\cite{DockerCoreEngineering2016,DockerInc2017} also uses ipvs for internal load balancing,
but it is also considered as part of Docker swarm infrastructure, 
and thus lacks the portability that our proposal aims to provide.
}

\begin{table}[h]
  \centering
  \begin{tabular}{|l|c|c|c|}
    \hline
    & \multicolumn{1}{c|}{Kubernetes} & \multicolumn{1}{c|}{Docker Swarm} & \multicolumn{1}{c|}{Mesos Marathon} \\ \hline
    Config file & YAML & YAML & JSON  \\ \hline
    Scheduling & Yes &  Yes &  Yes  \\ \hline
    Ingress routing & \begin{tabular}{c} Manual$^{*}$ \\ Cloud load balancer$^{**}$\end{tabular} & Manual$^{*}$ & Manual$^{*}$ \\ \hline
    Internal routing & iptables DNAT & \replaced[id=2nd]{IPVS}{ipvs} &  haproxy  \\ \hline
  \end{tabular}
  \centering\parbox[c]{0.9\columnwidth}{
    \footnotesize{$^{*}$Users are expected to set up a static route to one of the internal load balancers manually. \par
    $^{**}$Support for Cloud load balancer is only available in limited infrastructures including GCP, AWS, Azure and OpenStack. }
  }
  \caption{Container orchestrator comparison.}
  \centering\parbox[c]{0.9\columnwidth}{
    Important aspects of features as web application infrastructures are compared.
  }
%  \label{table:orchestrator_comparison}
\end{table}


\section{Ingress routing in Kubernetes}

\added[id=2nd]{
Nginx-ingress,
kube-keepalived,
metal-lb,
Static routing to, 
Kubernetes conventional.
}

\added[id=2nd]{
As far as load balancer containerization is concerned, the following related work has been identified:
Nginx-ingress\cite{Pleshakov2016,NginxInc2016} utilizes the ingress\cite{K8sIngress2017} capability of Kubernetes, 
to implement a containerized Nginx proxy as a load balancer. Nginx itself is famous as a high-performance web server program
that also has the functionality of a Layer-7 load balancer. Nginx is capable of handling Transport Layer Security(TLS) encryption, 
as well as Uniform Resource Identifier(URI) based switching. However, the flip side of Nginx is that it is much slower than Layer-4 switching.
We compared the performance between Nginx as a load balancer and our proposed load balancer in this paper.
%
Meanwhile, the kube-keepalived-vip\cite{Prashanth2016} project is trying to use Linux kernel's ipvs\cite{Zhang2000} 
load balancer capabilities by containerizing the keepalived\cite{ACassen2016}.
The kernel ipvs function is set up in the host OS's net namespaces and is shared among multiple web services,
as if it is part of the Kubernetes cluster infrastructure.
Our approach differs in that the ipvs rules are set up in container's net namespaces 
and function as a part of the web service container cluster itself.
The load balancers are configurable one by one, and are  movable with the cluster once the migration is needed.
The kube-keepalived-vip's approach lacks flexibility and portability whereas ours provide them.
}


\section{Load balancer tools in the container context}

\added[id=2nd]{
There are several other projects where efforts have been made to utilize ipvs in the context of container environment.
For example, GORB\cite{Sibiryov2015} and clusterf\cite{Aaltodoc:http://urn.fi/URN:NBN:fi:aalto-201611025433} are daemons 
that setup ipvs rules in the kernel inside the Docker container. 
They utilize running container information stored in key-value storages
like Core OS etcd\cite{CoreOSEtcd} and HashiCorp's Consul\cite{HashiCorpConsul}. 
Although these were usable to implement a containerized load balancer in our proposal, we did not use them, 
since Kubernetes ingress framework already provided the methods to retrieve running container information through standard API.
}

\section{Cloud load balancers}

\added[id=2nd]{
Maglev,
Kataran,
Ananta,
}

\added[id=2nd]{
Although this work is not seeking the best software load balancer for cloud.
This work is not seeking the better performance and scalability.
This work is seeking to guarantee inter opearability while keeping decent performance.
}

\begin{table}[h]
  \centering
  \begin{tabular}{|l|c|c|c|c|c|}
    \hline
    & \multicolumn{1}{c|}{OSS} & \multicolumn{1}{c|}{Container} & \multicolumn{1}{c|}{Redundancy} & \multicolumn{1}{c|}{Forwarding} & \multicolumn{1}{c|}{L3DSR} \\ \hline
    Maglev & No & No & ECMP & Flexible I/O layer & GRE  \\ \hline
    Ananta & No & No & ECMP & Windows  & GRE  \\ \hline
    Kataran& No & No & ECMP & XDP & IPIP  \\ \hline
    This work & Yes & Yes & ECMP & IPVS (XDP in future) & IPIP  \\ \hline
  \end{tabular}

  \begin{minipage}{0.9\columnwidth}
    \caption[Cloud load balancer comparison]{
    Cloud load balancer comparison.
    }   
    \label{tabl:cloud_lb}
  \end{minipage}
\end{table}

\section{Summary}

In this chapter the author highlighted relataed works of this thesis.
First and foremost
Container 

While the related work listed here is
The author regards his work is original since there is almost no other works that dealing with load balancer issue 
