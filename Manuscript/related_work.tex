
In this chapter, the author highlights the related work of this study.
The purpose of this research is to improve the portability of web applications by using container orchestrators as a common middleware.
Doing so will give users the freedom to migrate their services when there is a disaster, expand their businesses, and prevent vendor lock-ins, etc.
However, existing orchestrators fail to provide the same interfaces to web applications.
This is because none of them can fully automate the setup of routes for ingress traffic from the Internet, regardless of the base infrastructures.
Therefore, the author proposes a cluster of software load balancer containers for Kubernetes, which is deployed as a part of the web application clusters.

Here the author presents related work regarding the following subjects:
(1) Portability of web applications.
(2) Ingress routing in container orchestrators.
(3) Ingress routing in Kubernetes.

Additionally, there are several software load balancers for cloud environments, which try to differentiate their own cloud infrastructure by seeking the best performances.
On the other hand, the load balancer proposed in this study has a different purpose, which is to try to provide a load balancer common to any infrastructure by using standard OSS technologies.
Despite the difference in purposes, it is worthwhile comparing the technology components in order to assess if the proposed load balancer is state of the art.
There are also miscellaneous techniques, regarding the topic of this study, which should be presented.

Therefore, the author also presents related work regarding the following subjects:
(4) Cloud load balancer.
(5) Load balancer tools in the container context.


%% none of them has a standard way to set up the route for ingress traffic from the Internet automatically.
%% none of them fully supports features that automatically set up ingress routing in a redundant and scalable manner.
%% none of them provides full support for automatic set up of ingress traffic routing.
%% none of them has full support for automatically setting up routes for ingress traffic in a redundant and scalable manner.

%% This Chapter highlights related work, especially that dealing with container cluster migration,
%% software load balancer containerization, load balancer tools within the context of the container technology and scalable load balancer in the cloud providers.

%% It is important to make web application portable.
%% The author proposes to use container orchestrator as a common middleware.
%% The author focused on the problem regarding automation of ingress routing.
%% The author proposed architecture with a cluster of software load balancer containers.


\section{Portability of web applications}

\paragraph{\bf TOSCA:}


\paragraph{\bf Kubernetes federation:}

Kubernetes developers are trying to add federation \cite{K8sFederation2017} capability for handling situations 
where multiple Kubernetes clusters \footnote{The {\em Kubernetes cluster} refers to a server cluster 
controlled by the Kubernetes container management system, in this thesis.} 
are deployed on multiple cloud providers or on-premise data centers. 
Those Kubernetes clusters are managed by the Kubernetes federation API server (federation-apiserver).
Accoding to their explanation\cite{K8sFederation2017}, the federation capability provides the followings: 
\enquote{High Availability: By spreading load across clusters and auto configuring DNS servers and load balancers, federation minimises the impact of cluster failure.
Avoiding provider lock-in: By making it easier to migrate applications across clusters, federation prevents cluster provider lock-in.}
The author regards the federation capability very attractive.
However, how each Kubernetes cluster is run on different types of cloud providers
and/or on-premise data centers, especially when the load balancers of such environments are not supported by Kubernetes, 
seems beyond the scope of that project.
This thesis is mainly focused on how to provide a common load balancers to different type of infrastructures. 


\section{Ingress routing in container orchestrator}

Docker/Swarm
Mesos/Marathon
Kubernetes

The swarm mode of the Docker\cite{DockerCoreEngineering2016,DockerInc2017} also uses ipvs for internal load balancing,
but it is also considered as part of Docker swarm infrastructure, 
and thus lacks the portability that our proposal aims to provide.

\begin{table}[h]
  \centering
  \begin{tabular}{|l|c|c|c|}
    \hline
    & \multicolumn{1}{c|}{Kubernetes} & \multicolumn{1}{c|}{Docker Swarm} & \multicolumn{1}{c|}{Mesos Marathon} \\ \hline
    Config file & YAML & YAML & JSON  \\ \hline
    Scheduling & Yes &  Yes &  Yes  \\ \hline
    Ingress routing & \begin{tabular}{c} Manual$^{*}$ \\ Cloud load balancer$^{**}$\end{tabular} & Manual$^{*}$ & Manual$^{*}$ \\ \hline
    Internal routing & iptables DNAT & \replaced[id=2nd]{IPVS}{ipvs} &  haproxy  \\ \hline
  \end{tabular}
  \centering\parbox[c]{0.9\columnwidth}{
    \footnotesize{$^{*}$Users are expected to set up a static route to one of the internal load balancers manually. \par
    $^{**}$Support for Cloud load balancer is only available in limited infrastructures including GCP, AWS, Azure and OpenStack. }
  }
  \caption{Container orchestrator comparison.}
  \centering\parbox[c]{0.9\columnwidth}{
    Important aspects of features as web application infrastructures are compared.
  }
%  \label{table:orchestrator_comparison}
\end{table}


\section{Ingress routing in Kubernetes}

Nginx-ingress,
kube-keepalived,
metal-lb,
Static routing to, 
Kubernetes conventional.

As far as load balancer containerization is concerned, the following related work has been identified:
Nginx-ingress\cite{Pleshakov2016,NginxInc2016} utilizes the ingress\cite{K8sIngress2017} capability of Kubernetes, 
to implement a containerized Nginx proxy as a load balancer. Nginx itself is famous as a high-performance web server program
that also has the functionality of a Layer-7 load balancer. Nginx is capable of handling Transport Layer Security(TLS) encryption, 
as well as Uniform Resource Identifier(URI) based switching. However, the flip side of Nginx is that it is slower than Layer-4 switching.

Meanwhile, the kube-keepalived-vip\cite{Prashanth2016} project is trying to use Linux kernel's ipvs\cite{Zhang2000} 
load balancer capabilities by containerizing the keepalived\cite{ACassen2016}.
The kernel ipvs function is set up in the host OS's net namespaces and is shared among multiple web services,
as if it is part of the Kubernetes cluster infrastructure.

Our approach differs in that the ipvs rules are set up in container's net namespaces 
and function as a part of the web service container cluster itself.
The load balancers are configurable one by one, and are  movable with the cluster once the migration is needed.
The kube-keepalived-vip's approach lacks flexibility and portability whereas ours provide them.

\begin{table}[h]
  \centering
  \resizebox{\textwidth}{!}{
  \begin{tabular}{|l|c|c|c|c|c|}
    \hline
    & \multicolumn{1}{c|}{OSS} & \begin{tabular}{c}Container \\friendly\end{tabular} & \multicolumn{1}{c|}{Redundancy} & \multicolumn{1}{c|}{Forwarding} & \multicolumn{1}{c|}{L3DSR} \\ \hline
    Conventional & No & No & Static & iptables DNAT & No \\ \hline
    Nginx-ingress & Yes & Yes & No & nginx & No  \\ \hline
    kube-keepalived & Yes & Yes & VRRP & IPVS & No  \\ \hline
    metal-lb & Yes & Yes & ECMP & IPVS & No  \\ \hline
    This work & Yes & Yes & ECMP & IPVS$^{*}$ & IPinIP  \\ \hline
  \end{tabular}
  }

  \par\bigskip
  \begin{minipage}{0.9\columnwidth}
    \caption[Ingress routing techniques for Kubernetes]{
    Ingress routing techniques for Kubernetes.
    }   
    \label{tabl:k8s_lb}
  \end{minipage}

\end{table}

The proposed load balancer aims to be portable itself, while utilizing advanced technologies used in high performance cloud load balancers.

\section{Cloud load balancers}

As far as the cloud load balancers are concerned, two articles have been identified.
Google's Maglev \cite{eisenbud2016maglev} is a software load balancer used in Google Cloud Platform(GCP).
Maglev uses modern technologies including per flow ECMP and kernel bypass for user space packet processing.
Maglev serves as the GCP's load balancer that is used by the Kubernetes.
Maglev is not a product that users can use outside of GCP nor is an open source software, while the users need open source software load balancer that can run even in on-premise data centers.

Microsoft's Ananta \cite{patel2013ananta} is another software load balancer implementation using ECMP and windows network stack.
Ananta can be solely used in Microsoft's Azure cloud infrastructure\cite{patel2013ananta}.
The proposed load balancer by the author is different in that it is aimed to be used in every cloud provider and on-premise data centers.

Facebook's Katran \cite{2018katran} is an OSS software load balancer using Linux XDP tehnology.
Katran can also be used in ECMP redundant setups.
Although Katran is expected to have high performance level, nothing is published in the academic jurnals..

Although this work is not seeking the best software load balancer for cloud.
This work is not seeking the better performance and scalability.
This work is seeking to guarantee inter opearability while keeping decent performance.

The proposed load balancer in not yet competeing highest throughput.
The proposed load balancer in this thesis is more portable than cloud load balancers.

\begin{table}[h]
  \centering
  \resizebox{\textwidth}{!}{
  \begin{tabular}{|l|c|c|c|c|c|}
    \hline
    & \multicolumn{1}{c|}{OSS} & \begin{tabular}{c}Container \\friendly\end{tabular} & \multicolumn{1}{c|}{Redundancy} & \multicolumn{1}{c|}{Forwarding} & \multicolumn{1}{c|}{L3DSR} \\ \hline
    Maglev & No & No & ECMP & Flexible I/O layer & GRE  \\ \hline
    Ananta & No & No & ECMP & Windows Filtering Platform  & IPinIP  \\ \hline
    Katran& Yes & No & ECMP & XDP & IPinIP  \\ \hline
    This work & Yes & Yes & ECMP & IPVS (XDP in future) & IPinIP  \\ \hline
  \end{tabular}
  }

  \par\bigskip
  \begin{minipage}{0.9\columnwidth}
    \caption[Cloud load balancer comparison]{
    Cloud load balancer comparison.
    }   
    \label{tabl:cloud_lb}
  \end{minipage}
\end{table}


\section{Load balancer tools in the container context}

There are several other projects where efforts have been made to utilize ipvs in the context of container environment.
For example, GORB\cite{Sibiryov2015} and clusterf\cite{Aaltodoc:http://urn.fi/URN:NBN:fi:aalto-201611025433} are daemons 
that setup ipvs rules in the kernel inside the Docker container. 
They utilize running container information stored in key-value storages
like Core OS etcd\cite{CoreOSEtcd} and HashiCorp's Consul\cite{HashiCorpConsul}. 
Although these were usable to implement a containerized load balancer in our proposal, we did not use them, 
since Kubernetes ingress framework already provided the methods to retrieve running container information through standard API.


\section{Summary}

In this chapter the author highlighted relataed works of this thesis.
First and foremost
Container 

While the related work listed here is
The author regards his work is original since there is almost no other works that dealing with load balancer issue 
