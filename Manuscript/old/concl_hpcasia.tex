
\section{Conclusions}\label{Conclusions}

In this paper, we proposed a portable load balancer for the Kubernetes cluster systems 
that is aimed at facilitating migration of container clusters for web services.
We implemented a containerized software load balancer that is run by Kubernetes as a part of container cluster, 
using Linux kernel's IPVS, as a proof of concept.
In order to discuss the feasibility of the proposed load balancer, we built 
a Kubernetes cluster system and conducted performance measurements.
Our experimental results indicate that the IPVS based load balancer in container improves the portability of 
the Kubernetes cluster system while it shows the similar performance levels as the existing iptables DNAT based load balancer.
We also clarified that choosing the right operating modes of overlay networks is important for the performance of load balancers. 
For example, in the case of flannel, only the vxlan and udp backend operation modes could be used 
in the cloud environment, and the udp backend significantly degraded their performance.
Furthermore, we also learned that the distribution of packet processing among multiple CPUs was very important
to obtain the maximum performance levels from load balancers.
%

The limitations of this work that authors aware of include the followings: 
1) We have not discussed the load balancer redundancy. 
Routing traffic to one of the load balancers while keeping redundancy in the container environment is a complex issue,
because standard Layer 2 rendandacy protocols, e.g. VRRP or OSPF\cite{moy1997ospf} that uses multicast, can not be used in many cases.
Further more, providing uniform methods independent of various cloud environments and on-premise datacenter is much more difficult.   
2) Experiments are conducted only in a 1Gbps network environment.
The experimental results indicate the performance of IPVS may be limited by the network bandwidth, 1Gbps, in our experiments. 
Thus, experiments with the faster network setting, e.g. 10Gigabit ethernet, are needed to investigate the feasibility of the proposed load balancer.
3) We have not yet compared the performance level of proposed load balance with those of cloud provider's load balancers.
It shoud be fair to compare the performance of proposed load balancer with those of the combination of the cloud load balancer and the iptables DNAT. 
The authors leave these issues for future work and they will be discussed elsewhere.

