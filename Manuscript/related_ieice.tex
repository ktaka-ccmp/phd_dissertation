
This Chapter provides related work of this study.

\section{Related Work}\label{Related Work}

This section highlights related work, especially that dealing with container cluster migration, 
software load balancer containerization, load balancer tools within the context of the container technology and scalable load balancer in the cloud providers.

\paragraph{\bf Container cluster migration:}

Kubernetes developers are trying to add federation\cite{K8sFederation2017} capability for handling situations 
where multiple Kubernetes clusters\footnote{The {\em Kubernetes cluster} refers to a server cluster 
controlled by the Kubernetes container management system, in this paper.} 
are deployed on multiple cloud providers or on-premise data centers, 
and are managed via the Kubernetes federation API server (federation-apiserver). 
However, how each Kubernetes cluster is run on different types of cloud providers
and/or on-premise data centers, especially when the load balancers of such environments are not supported by Kubernetes, 
seems beyond the scope of that project. 
The main scope of this paper is to make Kubernetes usable in environments 
without supported load balancers by providing a containerized software load balancer.

\paragraph{\bf Software load balancer containerization:}
As far as load balancer containerization is concerned, the following related work has been identified:
Nginx-ingress\cite{Pleshakov2016,NginxInc2016} utilizes the ingress\cite{K8sIngress2017} capability of Kubernetes, 
to implement a containerized Nginx proxy as a load balancer. Nginx itself is famous as a high-performance web server program
that also has the functionality of a Layer-7 load balancer. Nginx is capable of handling Transport Layer Security(TLS) encryption, 
as well as Uniform Resource Identifier(URI) based switching. However, the flip side of Nginx is that it is much slower than Layer-4 switching.
We compared the performance between Nginx as a load balancer and our proposed load balancer in this paper.
%
Meanwhile, the kube-keepalived-vip\cite{Prashanth2016} project is trying to use Linux kernel's ipvs\cite{Zhang2000} 
load balancer capabilities by containerizing the keepalived\cite{ACassen2016}.
The kernel ipvs function is set up in the host OS's net namespaces and is shared among multiple web services,
as if it is part of the Kubernetes cluster infrastructure.
Our approach differs in that the ipvs rules are set up in container's net namespaces 
and function as a part of the web service container cluster itself.
The load balancers are configurable one by one, and are  movable with the cluster once the migration is needed.
The kube-keepalived-vip's approach lacks flexibility and portability whereas ours provide them.
%
The swarm mode of the Docker\cite{DockerCoreEngineering2016,DockerInc2017} also uses ipvs for internal load balancing,
but it is also considered as part of Docker swarm infrastructure, 
and thus lacks the portability that our proposal aims to provide.

\paragraph{\bf Load balancer tools in the container context:}
There are several other projects where efforts have been made to utilize ipvs in the context of container environment.
For example, GORB\cite{Sibiryov2015} and clusterf\cite{Aaltodoc:http://urn.fi/URN:NBN:fi:aalto-201611025433} are daemons 
that setup ipvs rules in the kernel inside the Docker container. 
They utilize running container information stored in key-value storages
like Core OS etcd\cite{CoreOSEtcd} and HashiCorp's Consul\cite{HashiCorpConsul}. 
Although these were usable to implement a containerized load balancer in our proposal, we did not use them, 
since Kubernetes ingress framework already provided the methods to retrieve running container information through standard API.

\paragraph{\bf Cloud load balancers:}

As far as the cloud load balancers are concerned, two articles have been identified.
Google's Maglev\cite{eisenbud2016maglev} is a software load balancer used in Google Cloud Platform(GCP).
Maglev uses modern technologies including per flow ECMP and kernel bypass for user space packet processing.
Maglev serves as the GCP's load balancer that is used by the Kubernetes.
Maglev is not a product that users can use outside of GCP nor is an open source software, while the users need open source software load balancer that is runnable even in on-premise data centers.
Microsoft's Ananta\cite{patel2013ananta} is another software load balancer implementation using ECMP and windows network stack.
Ananta can be solely used in Microsoft's Azure cloud infrastructure\cite{patel2013ananta}.
The proposed load balancer by the author is different in that it is aimed to be used in every cloud provider and on-premise data centers.




\cite{pellegrini2017preventing} "Preventing vendor lock-ins via an interoperable multi-cloud deployment approach"

\cite{felter2015updated} "An updated performance comparison of virtual machines and Linux containers"
