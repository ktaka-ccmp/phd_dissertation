\chapter*{Abstract}

Today, most of the people in the world can not spend a day without smartphones or PCs.
They use those devices to access services provided by web applications on the Internet.
These services include e-mail, social media, search engines, shopping site, etc., everything provided through the Internet.
As the services become indispensable part of the daily lives, operating web applications stably and swiftly becomes important day by day.
For example, those who provide these services need to be able to recover from a disaster, or start their web shopping site in other countries,
by migrating their web applications to different locations swiftly and safely.
For such purposes, providing a web application consisting of a cluster of Linux containers is a promising candidate, since Linux containers can be run on any Linux system regardless of the infrastructures.


A container orchestrator (also called container cluster management system) is a tool to simplify the management of a cluster of containers that are launched on multiple servers.
And it is expected to provide a uniform platform for container clusters, which also facilitates the migration of web applications consisting of container clusters.
However, none of the existing container orchestrators fully supports an automatic setup of ingress traffic routing from the Internet.
Users needed to set up a route for ingress traffic manually depending on the type of the infrastructure, every time they start a new web application.
The lack of this automation is one of the most critical problems for container orchestrators because without solving this problem, the migration of a web application will never be easy.

In this dissertation, the author addresses this problem by providing a portable software load balancer that is runnable on any infrastructure and capable of an automatic setup of the ingress traffic routing.
The author proposes a cluster of software load balancers in container for Kubernetes, that can be launched as a part of web applications.
It also supports an automatic setup of Equal Cost Multi Path(ECMP) routes to make multiple load balancers active, and thereby to provide redundancy and scalability.

The author has implemented a containerized software load balancer using Linux kernel's ipvs to prove the feasibility of the proposed load balancer architecture, and carried out performance measurements in the 1 Gbps network environment.
It has been shown that the proposed load balancers are runnable in an on-premise data center, Google Cloud Platform (GCP) and Amazon Web Service (AWS).
Therefore the proposed load balancers can be said to be portable.
%
The throughputs of a load balancer are dependent on the settings for multi-core packet processing and the setting for the overlay network.
It has been shown that the setting with as many CPU cores as possible for packet processing results in better performance.
It has been also shown that the backend mode for the overlay network without any packet encapsulation should be used for the best performance.
%
The throughput of the ipvs in container with Layer 3 Direct Server Return(L3DSR) setting has been about 1.5 times larger than that of existing iptables DNAT rules, which is prepared by Kubernetes's daemons as an  internal load balancer.
Therefore, the proposed load balancer has been proved to be portable without sacrificing the throughput of existing internal load balancer of the Kubernetes in 1 Gbps network environment.

The author also has implemented an automatic setup of the ECMP route for ingress traffic.
There, multiple load balancer containers are deployed, and each of them advertises itself as an active next hop of the IP for web application through Border Gateway Protocol(BGP).
The ECMP route makes the load balancers redundant and scalable since all the load balancer containers act as active.
The BGP helps automatic setup of the ECMP route.  
The BGP and ECMP are both standard protocols supported by most of the commercial router products.
%
The author verified that an ECMP route has been automatically created upon launch of a new load balancer container on the upstream router.
The update of the ECMP routing table was correct and quick enough, i.e., within 10 seconds, throughout 20 hours experiment.
The maximum performance levels of the cluster of load balancers have scaled linearly up to four times as the number of the load balancer containers has been increased to four of them.
The maximum aggregated throughput obtained through the experiment is 780k [req/sec], which is limited by the CPU performance of the benchmark client, and therefore can be improved using better hardware in the future experiment.
Therefore the author has proved that proposed load balancer has the capability of the automatic setup of ingress traffic in a redundant and scalable manner.

The author also extended the throughput measurement into the 10 Gbps network environment, in order to verify that proposed software load balancer is capable of providing needed throughput for 10 Gbps environment.
Although a single ipvs load balancer in the container can only provide 1/4 of required throughput, the parallelism of ECMP setups using more than four of them can provide enough throughput required in 10 Gbps environment.

The author has implemented a novel software load balancer using eXpress Data Path(XDP) technology for faster network and presented preliminary performance result.
The current implementation does not support multicore packet processing, and hence throughput is limited by the capability of single core processing performance.
However, the obtained throughput about 390K [req/sec] for the XDP load balancer is very promising.
The author estimates that about five of the software load balancer using this technology with 16 core packet processing can provide enough throughput in 100  Gbps environments in the future. 

The proposed load balancer has been verified to be portable while providing enough throughput in 10 Gbps environment.
The outcome of this study will benefit users who want to deploy their web services on any cloud provider where no scalable load balancer is provided, to achieve high scalability.
Moreover, the result of this study will potentially benefit users who want to use a group of different cloud providers and on-premise data centers across the globe seamlessly.
In other words, users will become being able to deploy a complex web service on aggregated computing resources on the earth, as if they were starting a single process on a single computer.


