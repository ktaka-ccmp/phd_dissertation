\chapter*{Abstract}

Linux containers have become very popular these days due to their lightweight nature and portability. 
Numerous web services are now deployed as clusters of containers. 
Kubernetes is a popular container management system that enables users to deploy such web services easily, and hence, 
it facilitates web service migration to the other side of the world.
However, since Kubernetes relies on external load balancers provided by cloud providers, 
it is difficult to use in environments where there are no supported load balancers.
This is particularly true for on-premise data centers, or for all but the largest cloud providers.
In this paper, we proposed a portable load balancer that was usable in any environment, and hence facilitated web services migration.
We implemented a containerized software load balancer that is run by Kubernetes as a part of container cluster, 
using Linux kernel's Internet Protocol Virtual Server(IPVS).
Then we compared the performance of our proposed load balancer with existing iptables Destination Network Address 
Translation (DNAT) and the Nginx load balancers.
During our experiments, we also clarified the importance of two network conditions to derive the best performance: 
the first was the choice of the overlay network operation mode, and the second was distributing packet processing to multiple cores.
The results indicated that our proposed IPVS load balancer improved portability of web services without sacrificing the performance.


Linux container technology and clusters of the containers are expected to make web services consisting of multiple web servers and a load balancer portable, and thus realize easy migration of web services across the different cloud providers and on-premise datacenters.
This prevents service to be locked-in a single cloud provider or a single location and enables users to meet their business needs, e.g., preparing for a natural disaster.
However existing container management systems lack the generic implementation to route the traffic from the internet into the web service consisting of container clusters.
For example, Kubernetes, which is one of the most popular container management systems, is heavily dependent on cloud load balancers. If users use unsupported cloud providers or on-premise datacenters, it is up to users to route the traffic into their cluster while keeping the redundancy and scalability.
This means that users could easily be locked-in the major cloud providers including GCP, AWS, and Azure.
In this paper, we propose an architecture for a group of containerized load balancers with ECMP redundancy.
We containerize Linux ipvs and exabgp, and then implement an experimental system using standard Linux boxes and open source software.
We also reveal that our proposed system properly route the traffics with redundancy.
Our proposed load balancers are usable even if the infrastructure does not have supported load balancers by Kubernetes and thus free users from lock-ins.

